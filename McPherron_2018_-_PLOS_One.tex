\documentclass[]{article}
\usepackage{lmodern}
\usepackage{amssymb,amsmath}
\usepackage{ifxetex,ifluatex}
\usepackage{fixltx2e} % provides \textsubscript
\ifnum 0\ifxetex 1\fi\ifluatex 1\fi=0 % if pdftex
  \usepackage[T1]{fontenc}
  \usepackage[utf8]{inputenc}
\else % if luatex or xelatex
  \ifxetex
    \usepackage{mathspec}
  \else
    \usepackage{fontspec}
  \fi
  \defaultfontfeatures{Ligatures=TeX,Scale=MatchLowercase}
\fi
% use upquote if available, for straight quotes in verbatim environments
\IfFileExists{upquote.sty}{\usepackage{upquote}}{}
% use microtype if available
\IfFileExists{microtype.sty}{%
\usepackage{microtype}
\UseMicrotypeSet[protrusion]{basicmath} % disable protrusion for tt fonts
}{}
\usepackage[margin=1in]{geometry}
\usepackage{hyperref}
\hypersetup{unicode=true,
            pdftitle={Additional statistical and graphical methods for analyzing site formation processes using artifact orientations},
            pdfborder={0 0 0},
            breaklinks=true}
\urlstyle{same}  % don't use monospace font for urls
\usepackage{longtable,booktabs}
\usepackage{graphicx,grffile}
\makeatletter
\def\maxwidth{\ifdim\Gin@nat@width>\linewidth\linewidth\else\Gin@nat@width\fi}
\def\maxheight{\ifdim\Gin@nat@height>\textheight\textheight\else\Gin@nat@height\fi}
\makeatother
% Scale images if necessary, so that they will not overflow the page
% margins by default, and it is still possible to overwrite the defaults
% using explicit options in \includegraphics[width, height, ...]{}
\setkeys{Gin}{width=\maxwidth,height=\maxheight,keepaspectratio}
\IfFileExists{parskip.sty}{%
\usepackage{parskip}
}{% else
\setlength{\parindent}{0pt}
\setlength{\parskip}{6pt plus 2pt minus 1pt}
}
\setlength{\emergencystretch}{3em}  % prevent overfull lines
\providecommand{\tightlist}{%
  \setlength{\itemsep}{0pt}\setlength{\parskip}{0pt}}
\setcounter{secnumdepth}{0}
% Redefines (sub)paragraphs to behave more like sections
\ifx\paragraph\undefined\else
\let\oldparagraph\paragraph
\renewcommand{\paragraph}[1]{\oldparagraph{#1}\mbox{}}
\fi
\ifx\subparagraph\undefined\else
\let\oldsubparagraph\subparagraph
\renewcommand{\subparagraph}[1]{\oldsubparagraph{#1}\mbox{}}
\fi

%%% Use protect on footnotes to avoid problems with footnotes in titles
\let\rmarkdownfootnote\footnote%
\def\footnote{\protect\rmarkdownfootnote}

%%% Change title format to be more compact
\usepackage{titling}

% Create subtitle command for use in maketitle
\newcommand{\subtitle}[1]{
  \posttitle{
    \begin{center}\large#1\end{center}
    }
}

\setlength{\droptitle}{-2em}

  \title{Additional statistical and graphical methods for analyzing site
formation processes using artifact orientations}
    \pretitle{\vspace{\droptitle}\centering\huge}
  \posttitle{\par}
    \author{}
    \preauthor{}\postauthor{}
      \predate{\centering\large\emph}
  \postdate{\par}
    \date{2019-03-13}

\usepackage{lineno}
\usepackage{setspace}\doublespacing

\begin{document}
\maketitle

Shannon P. McPherron \textsuperscript{1*}

\textsuperscript{1} Department of Human Evolution, Max Planck Institute
for Evolutionary Anthropology, DeutscherPlatz 6, Leipzig, D-04177,
GERMANY

\textsuperscript{*} Corresponding author

E-mail:
\href{mailto:mcpherron@eva.mpg.de}{\nolinkurl{mcpherron@eva.mpg.de}}

\pagebreak
\linenumbers

\textbf{Abstract}

The 3D orientation of clasts within a deposit are known to be
informative on processes that formed that deposit. In archaeological
sites, a portion of the clasts in the deposit are introduced by
non-geological processes, and these are typically systematically
recorded in archaeological excavations with total stations. By recording
a second point on elongated clasts it is possible to quickly and
precisely capture their orientation. The statistical and graphical
techniques for analyzing these data are well published, and there is a
growing set of actualistic and archaeological comparative data to help
with the interpretation of the documented patterns. This paper advances
this area of research in presenting methods to address some shortcomings
in current methodologies. First, a method for calculating confidence
intervals on orientation statistics is presented to help address the
question of how many objects are needed to assess the formation of a
deposit based on orientations. Second, a method for assessing the
probability that two assemblages have different orientations is
presented based on permutations testing. This method differs from
existing ones in that it considers three-dimensional orientations rather
than working separately with the two-dimensional bearing and plunge
components. Third, a method is presented to examine spatial variability
in orientations based on a moving windows approach. The raw data plus
the R code to build this document and to implement these methods plus
those already described by McPherron {[}1{]} are included to help
further their use in assessing archaeological site formation processes.

\textbf{Keywords}

Site Formation, Artifact orientations, Benn Diagram, Permutations Tests,
Confidence Intervals, Resampling

\textbf{Introduction}

The 3D orientation of clasts within a deposit are known to be
informative on processes that formed that deposit. In archaeological
sites, a portion of the clasts in the deposit are introduced by
non-geological processes, meaning typically through human behaviors but
also in some instances from other biological agents such as carnivores.
The find location of these non-geogenic clasts are typically
systematically recorded in archaeological excavations, and with the
introduction of total stations for precise piece proveniencing {[}2{]}
it became possible to record an additional point for elongated finds to
quickly and precisely capture their orientation in the normal course of
documenting the excavation {[}1,3{]}. Thus sometimes relatively
substantial and informative data sets for the study of post-depositional
formation processes can be amassed with little additional effort. The
statistical and graphical techniques for analyzing these data are well
published {[}1,4,5{]}, and there is a growing set of actualistic and
archaeological comparative data to help with the interpretation of the
documented patterns {[}3,6--26{]}.

This paper advances this area of research in presenting several new
methodologies for analyzing orientation data. These methods are designed
to address several shortcomings in current methodologies. First, one
difficulty in orientations analysis is in knowing how large a sample is
needed to have a reliable or, more importantly, informative measure.
Neudorf et al. {[}27{]} use eigenstatistics (see below) to argue that
after approximately 35 measurements variability is minimized enough to
achieve robust results. Lenoble and Bertran {[}5{]} use three data sets
with unknown orientation characteristics to show that the Vector
Magnitude (L) statistic (a measure of the strength of the dominant
bearing in an assemblage) stabilizes after about 40 or 50 artifacts, and
they argue that with this size of sample highly linear assemblages can
be distinguished from other kinds of assemblages. In other words, the
required sample size is in part a function of the distribution of
orientations in the assemblage. This point is emphasized by Ringrose and
Benn {[}28,29{]} who also argue that 50 observations are required to
reach a conclusion about the orientations. To show this and to provide a
solution to the sample size problem, they propose a method for
calculating confidence intervals for Benn diagrams {[}4{]}. While Benn
diagrams have proven particularly useful for representing and comparing
three dimensional orientations {[}1,5{]}, in a Benn diagram assemblages
are represented by a single point with no indication of the associated
uncertainty {[}28{]}. Thus to address these issues here, rather than
arriving at specific recommendations for sample sizes, a method for
calculating 95\% confidence intervals on Benn diagrams is presented
based on the work of Ringrose and Benn {[}28--30{]} and on what has been
proposed by Weaver and Steele {[}31,32{]} for assessing mortality
profiles in faunal assemblages also ternary diagrams. With confidence
intervals, the interaction between the underlying orientation tendencies
and sample size are exposed. These results confirm that the size of the
confidence interval for a given sample size depends very much on the
structure of the data or, in other words, the type of post-depositional
disturbance that may have occurred. Confidence intervals also allow for
a better assessment of whether and what kinds of post-depositional
disturbances may have affected the deposit by showing potential overlap
between the uncertainty in the observed data and the convex hulls of
sets of points known already to represent various kinds of processes
{[}5{]}.

Second, another difficulty in orientations analysis is in knowing just
how similar or different assemblages are in their three dimensional
orientation structure (i.e.~after having computed eigenvalues and
plotting these on a Benn diagram){[}28{]}. Though there are tests for
comparing bearing angles (e.g.~Rayleigh test) or plunge angles
(e.g.~unpaired t-test), to my knowledge there exists no recommended test
for comparing the two together. The above mentioned confidence intervals
on the Benn diagrams do help give some indication of the degree to which
assemblages may differ {[}27,29,33{]}, and here an additional option of
applying a permutations analysis to the Euclidean distance between the
Benn indices (elongation and isotropy) is presented to address this
problem.

Third, there is a lack of tools for exploring spatial variability in
orientations. If there are reasons to segment an assemblage into
spatially distinct sets, then the above techniques allow them to be
compared. However, when there are no a priori reasons to spatially
segment an assemblage, a method is needed to examine whether there is,
nevertheless, spatial patterning in the data. To address this, a method
for calculating and plotting nearest neighbor samples and, in effect,
moving averages across an assemblage from a particular stratigraphic
unit is presented. This method allows spatial patterning across an
assemblage to be visualized in Benn space. When patterns are detected,
the assemblage can then be spatially segmented and tested following the
above mentioned techniques.

Additionally, to further advance the analysis of artifact orientations,
all of the software used in this paper to make the standard {[}1{]} and
the newly presented statistical and graphical techniques are available
in the Supplemental Information. This code, written in R {[}34{]},
addresses an additional shortcoming, namely easy access to tools needed
to do this type of research.

\textbf{Materials}

Three data sets are used here: a simulated one, a previously published
one, and a recently excavated one. The simulated data set consists of
varying proportions of a perfectly planar assemblage, meaning objects
laying on a surface and oriented in all directions on this surface, and
a nearly perfectly linear assemblage, meaning objects all oriented in
exactly the same direction (see Lenoble and Bertran {[}5{]} for
illustration of these terms). To avoid having these assemblages fixed to
the base line of the Benn diagram (see methods below) and to make these
simulated assemblages more like archaeological ones, a uniformly random
dispersion from horizontal of up to 10 degrees has been introduced into
these data (lifting them slightly towards the isotropic pole of the Benn
diagram). Because a perfectly linear assemblage would not have
variability that could be meaningfully resampled, the linear assemblages
are drawn from horizontal angles uniformly, randomly distributed from
160 to 200 degrees (20 degrees to either side of grid south) and with
the same vertical dispersion of 10 degrees. From these two assemblage
types, mixed assemblages are created in varying proportions of 0\%,
20\%, 40\%, 60\% and 80\% linearly oriented objects (the rest being
drawn from the planar assemblage) and with varying assemblage sizes in
increments of 20 from 30 to 150. The R code to create, analyze and
present these simulated assemblages is provided in the Supplemental
Information.

To help interpret the results presented here, the comparative
orientations data from natural slope deposits published by Lenoble and
Bertran {[}5,35{]} were digitized using the open source program
WinPlotDigitizer Version 3.8. These data for debris flow, for runoff on
steep and shallow slopes, and for solifluction are used here to provide
some indication of the range of Benn values for these formation
processes. This in turn helps to better understand whether a particular
95\% confidence interval for an assemblage includes the possibility of
more than one interpretation for the formation of the deposit. Note that
for solifluction, one extreme outlier in the published data set is
excluded here because it falls so close to the planar position that this
must represent a type of solifluction quite distinct from the remaining
data points as to make interpretation less meaningful (note that
{[}36{]} also drops this point from the solifluction comparative data
set).

The excavated data set comes from recent excavations, led by Alain Turq,
at the site of La Ferrassie (Table 1) {[}37,38{]}. The 3m La Ferrassie
stratigraphic sequence consists of multiple Middle and Upper Paleolithic
levels with bone and stone artifacts exposed over several square meters.
As summarized in {[}37,38{]}, the sediments are fluvially deposited
sands and gravels overlain by slope deposits. The transition between
these two depositional regimes occurs in Layer 3, and Layer 2 shows
evidence of deposition and subsequent alteration under a cold climate
regime. In excavating these deposits, all bones and stones larger than
2.5 cm were piece provenienced with a total station (typically with 5
second precision). All such elongated objects were recorded with two
points, one at each end of the long axis (see {[}1{]}). What constituted
elongated was not numerically defined. Rather excavators were initially
instructed on examples of elongated objects and the reasons for
recording and analyzing two points. The site supervisor helped with
initial determinations and was always available to provide some control.
However, in our experience, excavators quickly develop a sense of which
objects merit two points. For a subset of the stone artifacts (the
complete flakes and tools) from the Middle Paleolithic layers, the
elongation can be calculated from the length and width measurements
taken during subsequent stone tool analysis. Length here is recorded
from the point of percussion to the furthest point, and width is
recorded at the midpoint of this line and perpendicular to it. The
median length of the long axis of all objects is calculated based on the
Euclidean distance between the two total station coordinates for each
object. Both of these measures, elongation and length, can be used to
filter from analysis artifacts that are perhaps not elongated enough to
have had a clear principal axis or artifacts that are too small when
there are some doubts about the quality of the recorded data or when,
for instance, examining the effect of post-depositional processes on
different size or shape classes. Here these values are reported, but
they were not used to filter the data set.

\begin{longtable}[]{@{}lrrrlll@{}}
\caption{La Ferrassie two-shot sample characteristics by layer.
Elongation mean and standard deviation are calculated on a sub-sample of
lithics with length and width measurements made with calipers (see text
for how these measurements were made). Median length is calculated based
on the two total station coordinates for the object.}\tabularnewline
\toprule
& Bones & Lithics & Length (Median) & Elongation (n) & Elongation (Mean)
& Elongation (SD)\tabularnewline
\midrule
\endfirsthead
\toprule
& Bones & Lithics & Length (Median) & Elongation (n) & Elongation (Mean)
& Elongation (SD)\tabularnewline
\midrule
\endhead
7B & 65 & 22 & 0.039 & & &\tabularnewline
7A & 460 & 159 & 0.042 & & &\tabularnewline
6 & 25 & 53 & 0.046 & & &\tabularnewline
5B & 195 & 36 & 0.049 & 32 & 2.35 & 0.66\tabularnewline
5A & 171 & 15 & 0.055 & 8 & 1.51 & 0.88\tabularnewline
4 & 339 & 37 & 0.058 & 29 & 2.29 & 0.84\tabularnewline
3 & 242 & 33 & 0.051 & 26 & 2.22 & 0.87\tabularnewline
2 & 189 & 14 & 0.058 & 13 & 2.49 & 0.63\tabularnewline
1 & 23 & 9 & 0.063 & 9 & 1.83 & 0.94\tabularnewline
\bottomrule
\end{longtable}

The La Ferrassie data summarized in Table 1 and presented below are
provided in the Supplementary Information as an R data object.

\textbf{Methods}

The orientation of an object's long (a-) axis line can be described by
the geological terms bearing or trend and plunge {[}39,40{]}. Bearing is
the horizontal angle of the long axis line relative to some geographic
or arbitrary north, and the plunge is the vertical angle of the line
relative to the horizontal plane. An object with a plunge angle of 0 is
lying level on a horizontal plane, and an object with a plunge angle of
90 is perpendicular to the horizontal or straight up and down. Normally
bearing angles range from 0 to \textless{}360 degrees. An object with a
bearing of 0 degrees is pointed due north (positive y-axis) in the site
grid, and an object with a bearing of 90 is pointed due east (positive
x-axis) in the site grid. However, without the added information coming
from the plunge angle, an object with a bearing of 0 is
indistinguishable from its complement (180 degrees). By considering also
the plunge angle, an object can then be said to be plunging towards the
north (bearing is 0 degrees) versus to the south (bearing 180 degrees).
In what is presented here, bearing angles vary across 360 degrees and
are visualized as plunging in the bearing direction using Schmidt lower
hemisphere diagrams with a superimposed Rose diagram (see {[}1{]}). So
for clarity, here the term orientation refers to the general alignment
of an object in three dimensional space and the terms bearing and
plunge, as just defined, are used to describe the respective horizontal
and vertical components as necessary. Note that bearing is periodic or
circular and as such requires special statistical treatment whereas
plunge is not circular and can be treated with ordinary descriptive
statistics. Here, however, the two are treated together as a vector
describing the orientation of the object.

Some descriptive statistics and visual methods for treating orientation
data are presented in McPherron {[}1{]} and citations within. One of
these methods, specifically the use of Benn ratios, is the basis of the
new techniques presented here. Benn {[}4{]} showed that two ratios,
elongation and isotropy, calculated from computed eigenvalues {[}41{]}
and plotted on a kind of modified ternary diagram are useful for
summarizing assemblage orientations. Eigenvalues represent the degree of
clustering around three mutually orthogonal eigenvectors. They sum to 1,
and the first eigenvalue is the largest, followed by the second and then
the third. Thus how this sum is distributed across the three eigenvalues
says something about the orientations. The first eigenvalue represents
the maximum clustering in the data. A high value for this eigenvalue and
a low value for the other two indicates linear orientations (all objects
pointed generally in the same direction). When the first two eigenvalues
have roughly equal values and the third a lower value, then the
orientations are planar (objects randomly oriented on a plane). When all
three eigenvalues are roughly equal then there is no preferred
orientation, and the data are considered to be isotropic (randomly
oriented in three dimensions). The elongation and isotropy ratios
capture this patterning. The elongation ratio is computed as 1 minus the
ratio of the second eigenvalue to the first. Thus when the first
eigenvalue is large in comparison to the second, their inverse ratio
becomes smaller and the elongation ratio approaches 1. When the two
values are roughly equal, the inverse ratio approaches 1 and the
elongation ratio then approaches 0. The isotropy ratio then captures the
relationship between the third eigenvalue and the first (again expressed
as ratio of the two values but this time without subtracting from 1).
Now when the first eigenvalue is large in comparison to the third, the
inverse ratio of the two approaches zero, and to the contrary when they
are roughly equal, the isotropy ratio approaches 1. Thus a perfectly
isotropic assemblage will have an elongation ratio of 0 and an isotropy
ratio of 1. A perfectly linear assemblage will have an elongation ratio
of 1 and an isotropy ratio of 0. When both ratios are 0, the assemblage
is planar. This approach has the advantage of simultaneously considering
the bearing and the plunge to describe orientations, and it is well
suited to the kind of a-axis orientation data that the total station
recording method provides.

The code to implement the methods described below is written in R
{[}34{]} by the author. This code makes use of the CircStats {[}42{]},
colorspace {[}43--45{]}, KernSmooth {[}46{]}, fields {[}47{]}, spatial
{[}48{]}, and tiff {[}49{]} packages. This document (included in the
Supplemental Information as well in rmarkdown format) was prepared using
the dplyr {[}50{]}, reshape {[}51{]}, knitr {[}52--54{]}, xtable
{[}55{]} and knitcitations {[}56{]} packages.

To calculate confidence intervals in Benn space (elongation vs.~isotropy
indices), the Benn space is divided into 100 intervals (representing .01
increments of each index) or a square matrix of 10,000 individual cells.
The original assemblage is then resampled with replacement repeatedly,
and with each resampled assemblage Benn indices are calculated and
totaled by the respective cell. In the end, the 100 x 100 matrix is
examined to determine what count value represents the threshold at the
appropriate probability value (typically .95). At this threshold, the
sum of the cells with counts above this value account for, for instance,
95\% of the resampled assemblages, and the remaining cells lie outside
the 95\% confidence interval. The contourLines function (grDevices) is
then used to calculate the appropriate contour lines through this matrix
along the two Benn indices, and the results are plotted. The choice of a
100 x 100 grid is arbitrary and intended only to give a satisfactory
visual result. Resampling is repeated 10,000 times (cf. {[}28{]}).
Again, this is an arbitrary decision intended to trade computational
time against visual results. Resampling fewer times simply results in
rougher contour lines often also with small, isolated pockets of
probability at the edges of the central 95\% contour.

While the method proposed here is based on one presented by Ringrose and
Benn {[}28--30{]}, it differs from theirs in the following way. Ringrose
and Benn noted that when there are eigenvalues of similar magnitude
(e.g.~the first two eigenvalues for planar fabrics and all three for
isotropic fabrics), resampling can result in these eigenvalues swapping
places such that, for instance, the first eigenvector becomes the second
and vice versa. What this does, in turn, is reduce the amount of
variability that can occur in the resampling as eigenvectors and their
associated eigenvalues that would have otherwise moved further apart are
swapped thereby decreasing their apparent differences. In addition, when
samples plot close to the left edge of the Benn diagram, e.g.~when the
fabric is planar and the first two eigenvalues are nearly equal in
magnitude, resampling will less frequently result in eigenvalues of more
similar magnitude and more frequently result in less similar
eigenvalues. This then will impact the shape of the 95\% confidence
interval by making the resampled points drift to the right of the
original sample point (see {[}28{]} for a complete explanation). While
noting that drift is to some extent unavoidable in the eigenvalue
approach, Ringrose and Benn{[}28{]} propose retaining the original order
of the eigenvalues (thereby violating the rule that they decrease in
magnitude) by inspecting the eigenvectors to note when they have been
swapped, re-ordering the eigenvectors and associated eigenvalues to
reflect the original order of the eigenvectors, and expanding the simple
one panel Benn diagram into a six panel diagram to account for the new
possible arrangements of the eigenvalues. In what is presented here, the
eigenvalues are allowed to swap positions, and panels are not added to
the Benn diagram. This decision was made for three reasons. First, an
informal assessment of the impact of eigenvector swapping on the 95\%
confidence intervals showed that practically speaking its impact on the
location of the confidence interval on the main Benn diagram was
relatively minor. Not adding additional panels means that the resampled
assemblage often plots to the left of the 95\% confidence interval
rather than at its center, but, second, the proposed multiple panel
diagram solution has not been widely accepted (but see {[}33{]} and
{[}27{]}) meaing that, third, nearly all of the comparative data used to
interpret the plots uses the standard Benn diagram. Thus while the
Ringrose and Benn proposal is preferable from a statistical perspective,
it is not practical at this time and, importantly, not applying it does
not appear to change the interpretation of deposit formation based on
fabric orientation. Nevertheless, the code to compute the resampled
eigenvalues following Ringrose and Benn is included in the attached
files.

To assess the likelihood that two assemblages come from the same
population (i.e.~the null hypothesis that they have the same
orientations cannot be rejected), a permutations test is applied. To do
this, first, the Euclidean distance between the two assemblages in Benn
space is calculated (i.e.~the distance between where they plot using the
elongation and isotropy indices). Next, the two assemblages are
combined, and two new assemblages are drawn randomly from this combined
assemblage without replacement and with sample sizes equal to the
original two assemblages respectively. The Euclidean distance between
the Benn indices of these resampled assemblages is calculated. Finally,
this step is repeated a set number of times to create a distribution of
expected distances if the two assemblages represent a random
partitioning of a single population into two assemblage groups. Once
this is done, the originally calculated actual distance between the two
assemblages is compared to the distribution of expected values to
calculate its probability. This is a one-sided probability because we
are interested in instances where the distance is more (not less) than
we would expect based on resampling.

To view spatial variation in object orientations, Benn indices are
calculated for every object on a sample made of neighboring objects.
What counts as neighboring is arbitrary and could be based on distance
or sample size. The two are typically directly related, but here the
nearest 40 objects, rather than a fixed distance, are used. This value
was chosen arbitrarily (and can be varied) to have enough objects in
each sample to reasonably compute Benn indices (while confidence
intervals could be computed for each of these samples, it is not obvious
how to visualize this). One drawback to this approach is that, in some
cases, depending on how the densities of objects varies spatially,
objects from fairly far away (or outside what one would consider to be
the local area of interest for that object) might get included in the
sample. To protect against this, a maximum distance threshold is
included in the calculation as an option. This option is not applied
here because in general the spatial arrangement of the objects is
suitable and because it can result in samples of varying sizes. Next,
the space within the Benn diagram is given a color coding that varies
from green to red from the planar to the linear poles respectively
(changes in the elongation index) and varies in saturation (amount of
white mixed into the color) based on the distance from the isotropic
pole (isotropy index). Each local assemblage is plotted on the Benn
diagram in black but the corresponding color is used to plot the object
on a plan view of the entire assemblage. In this way, the Benn diagram
then becomes both a key to reading the spatial plot and itself a
representation of variability within the assemblage not unlike the 95\%
confidence intervals but in this case spatial structured. A large
variation in the distribution of points on the Benn diagram will then
suggest that spatial differences exist within the assemblage, and these
can be further examined with reference to the color coded plan views.
This method is purely exploratory and offers no statistical tests of
significance or descriptive statistics of the spatial patterning.
However, this can be achieved easily by then dividing the assemblage
into spatially defined groups based on inspecting the results visually
and by then calculating standard orientation statistics on these groups
(including the above described permutations test for differences between
groups).

\textbf{Results}

The effect of sample size and of varying proportions of planar and
linear objects on the 95\% confidence intervals of the Benn ratios and
on the range of possible interpretations is demonstrated in Fig 1. In
Fig 2, the Benn ratios and 95\% confidence intervals are shown for all
the main layers from the La Ferrassie excavations. Fig 3 and Table 2
show the results of pair-wise permutations test between the Benn ratios
for each layer combination. Finally, Figs 4-8 show how the Benn ratios
vary spatial within each layer, and Fig 9 shows how changing the
sampling size window impacts the results for La Ferrassie Layer 2.

\textbf{Fig 1. Confidence intervals on simulated data showing the effect
of sample size with varying proportions of planar and linear data. The
black dot in each case shows the actual calculated value for that
assemblage. Each column of figures has the same sample size. Each row of
figures has the same proportional mix of planar and linear samples. E1,
E2 and E3 are the first, second and third eigenvalues respectively.
Comparative deposition data from {[}5{]}.}

\textbf{Fig 2. Benn diagrams for each La Ferrassie layers. The dot
indicates the calculated values for that layer. The confidence intervals
represent a probability of 0.95 based on 10 times resampling.}

\textbf{Fig 3. Benn diagram for all La Ferrassie layers. Connecting
lines indicate layers that do not differ statistically at the p=0.05
level based on a permutations test with 10 times resampling (see Table
2).}

\begin{table}[ht]
\centering
\begin{tabular}{rrrrrrrrrr}
  \hline
 & 1 & 2 & 3 & 4 & 5A & 5B & 6 & 7A & 7B \\ 
  \hline
1 &  & 0.00 & 0.80 & 0.10 & 0.00 & 0.10 & 0.50 & 0.30 & 0.80 \\ 
  2 &  &  & 0.00 & 0.00 & 0.00 & 0.00 & 0.00 & 0.00 & 0.00 \\ 
  3 &  &  &  & 0.10 & 0.10 & 0.10 & 0.90 & 0.50 & 0.40 \\ 
  4 &  &  &  &  & 0.30 & 0.80 & 0.20 & 0.00 & 0.00 \\ 
  5A &  &  &  &  &  & 0.50 & 0.10 & 0.00 & 0.00 \\ 
  5B &  &  &  &  &  &  & 0.10 & 0.00 & 0.00 \\ 
  6 &  &  &  &  &  &  &  & 0.90 & 0.00 \\ 
  7A &  &  &  &  &  &  &  &  & 0.00 \\ 
  7B &  &  &  &  &  &  &  &  &  \\ 
   \hline
\end{tabular}
\caption{Probabilities resulting from permutation test results of layer pair comparison.  Results based on 10  times resampling.  Probabilities > 0.05 are plotted with connecting lines in Fig 3.} 
\end{table}

\textbf{Figs 4-8. Artifact orientations for La Ferrassie. The plan view
shows all two-shot bone and stone artifacts. The underlying image is a
georeference orthophoto extracted from a structure from motion model.
With the exception of Layer 1, the plan view points are color-coded
using Benn statistics computed on the 40 closest artifacts. The color
key is shown in the Benn diagram where these same points are also
plotted as black dots. Each figure also includes a Schmidt lower
hemisphere plot summarizing bearing and plunge angles with a
superimposed Rose diagram summarizing only the bearing angle
distributions {[}1{]}. Below this is a circular histogram of plunge
angles. Though the sample size in Layer 1 is not large enough for
spatial color coding, the layer is included here for completeness of the
presentation of La Ferrassie artifact orientations.}

\textbf{Fig 9. Artifact orientations for La Ferrassie Layer 2 with
varying sampling windows (see also Figs 4-8 for the color key and for a
sampling window of 40). With a small sampling window more small scale
variation is apparent but it is less clear whether this is meaningful
variability or sampling bias. With a large sampling window, the spatial
variability in this layer is less apparent as the color of each
individual point approaches the assemblage average.}

\textbf{Discussion}

The results of the resampling of simulated assemblages varying in their
proportion of linear and planar objects demonstrates that the size of
the confidence intervals vary in two predictable ways. First, as the
sample size increases, the size of the confidence interval decreases
(reading across each row of Fig 1). Second, the size of the confidence
interval depends on the structure of the underlying sample. If the
sample is more linear, then the confidence interval will be smaller
(reading down each column of Fig 1). Conversely, planar assemblages have
larger confidence intervals. What this means practically is that, for
instance, by the time 50 objects are measured, if the sample shows a
tendency towards linear orientations, then one can be fairly confident
that indeed the sample comes from an assemblage of linearly aligned
objects. However, conversely, if the assemblage plots more towards the
planar pole on the Benn diagram, then there will still be a large
uncertainty in the result. Assemblages that initially appear to be good
(i.e.~close the planar pole) may in fact show patterns more indicative
of post depositional disturbances after a larger sample is collected and
vice versa. Though it was not modeled here, increased isotropy would
behave like planar assemblages in that low sample sizes will have larger
uncertainties than either linear or planar assemblages of the same size.
In other words, with more variability in the sample, there is less
confidence in whether this variability is a function of sampling bias or
representative of the true underlying pattern. As a result, there can be
no one recommendation for a minimum sample size for artifact
orientations. The best practice recommendation is to collect orientation
data until the size of the confidence interval is sufficiently small
that the range of possible interpretations for the formation of that
deposit is reduced. If it is important to test whether one assemblage is
distinct from another (see below) a large sample may be required, and in
any case, orientations should be reported with confidence intervals.

Note that when orientation data are collected with a total station and
data collector, the code provided here makes it practical to process the
data on a daily basis and adjust excavation strategies accordingly. So,
for instance, in deposits that are low in artifact densities, a decision
can be made to record small natural clasts, often times not normally
recorded during excavation, to increase sample sizes. While perhaps in
some instances natural clasts and artifacts may have different
orientations and while this is in itself a potentially interesting line
of investigation which can be tested using the permutations technique,
at least the combined sample could give a greater confidence on the
formation of a particular deposit. Once this is achieved, the excavation
strategy can then be modified to return to the more standard approach of
not documenting in detail small natural clasts.

The results for the La Ferrassie layers indicate that with the 95\%
confidence intervals only Layers 4 and 5 plot close to the planar
portion of the Benn diagram with very little overlap (i.e.~a low
probability) with comparative data for depositional processes such as
solifluction, run-off or debris flow. Especially Layer 1 but also Layer
6 have low sample sizes and this is reflected by their large confidence
intervals. In both instances, it is not clear from the sample we have
whether the objects in this layer have moved after deposition or not.
Layer 1 could represent almost any kind of deposit, based on the
orientations only, whereas Layer 6 is more likely either undisturbed or
disturbed somewhat by objects shifting downslope. Layers 3 and 7 also
show some post-deposition alteration consistent with downslope movement.
The position of Layer 2 is consistent with either solifluction or
movement on a steep slope. Note, in keeping with the results of the
simulated data sets, the confidence interval is much smaller than seen
for Layers 5a and 5b despite similar sample sizes.

The results presented here also show that permutations test are an
effective way to compare the orientations of two assemblages in three
dimensions. This is useful because Benn diagrams, which summarize both
the bearing and plunge information in an assemblage of oriented objects,
have been shown to be such a useful tool for interpreting site formation
processes. Here the La Ferrassie data set is used to illustrate the
method. The results follow the 95\% confidence interval results and
show, for instance, that indeed Layer 2 is unlike any of the other
layers and that Layers 4, 5A and 5B are quite similar to each other but
different from the underlying Layer 3. However, in addition to comparing
whole layers in a site like La Ferrassie, for instance, the method gives
a useful tool for comparing sub-assemblages. With permutations testing,
the likelihood that natural stones and artifacts or that bones and
stones or that small and large objects have the same orientations can be
assessed. Permutations testing returns a specific probability (based on
randomized, repeated resampling) that can be used for null hypothesis
testing. However, it can also be used simply as a relative measure of
the similarity of two assemblages.

Finally, based on its application to the La Ferrassie data set, the
graphical and statistical technique applied here to presenting spatial
variation in object orientations appears to be effective. For instance,
though the above results showed that Layer 2 had been
post-depositionally altered, the spatial results show that consistent
differences are clearly visible across the layer with one portion
showing more heavily aligned artifacts than the other. In fact, this
difference corresponds to a channel like feature visible during
excavations and in the remaining profiles and that likely represents a
solifluction lobe. In other layers of La Ferrassie, especially Layers 7A
and 7B, differences can be seen in the orientations of objects in the
grid north section of the site. This portion of the excavation area is
near or immediately adjacent to the cliff face and likely represents
some type of ``wall effect'' common in cave sites. So while the
assemblage averages with 95\% confidence intervals for Layer 7 indicate
some post-depositional alteration, the spatial data reveal a more
complex picture with parts of the layer looking intact while other parts
have clearly suffered some alterations. Thus this method allows for a
more nuanced view of site formation processes, especially when combined
with permutations testing to help quantify the magnitude of observed
differences.

One important issue with this approach is the selection of the size of
the sample window used to calculate the Benn statistics for each object.
On the one hand, small sample sizes may be more sensitive to small scale
variations in object orientations, and this could be important for
interpreting the formation of a deposit; however, they will also be
subject to random variations related to biased sampling (see above).
Further, the resampling and confidence intervals approach implimented
here demonstrates that the sensitivity of the size of the sampling
window to random variation is dependent on the structure of the
orientations. On the other hand, a large sampling window may obscure
important patterning. Thus there is no simple solution. The best
practice recommendation in this instance is to try windows of different
scales and assess the results in each case (Fig 9). This approach is
similar to what is recommended in other types of spatial analysis, and
it may in fact be useful to apply some additional spatial statistics to
quantitatively assess whether the resulting spatial patterns at a
particular scale or sample size depart from expectations under a random
model.

\textbf{Conclusion}

The goal of this paper has been to provide some possible solutions to
methodological issues that exist with the analysis of object
orientations. The issues addressed here are what sample size is required
for secure interpretation of site formation processes, how to make
pair-wise probability based comparisons of assemblage orientations in
three dimensions, and how to explore and represent spatial variability
in orientations within an assemblage. The methods presented here are
proposals and consist respectively of the use of confidence intervals to
assess sample sizes, permutation testing to assess assemblage
differences, and moving sampling windows to visualize spatial
variability. As for the latter method, the application of spatial
statistics to the observed spatial patterns is an obvious next step that
will help bring additional statistical support for the interpretations
of observed patterning.

R code to implement each of these methods as well as the complete
example data set from La Ferrassie are included here with these
proposals. It is clear that as researchers increasingly rely on results
coming from program code, a proper evaluation of the results requires
not just the data but also the code {[}57{]}. Thus, here the general
routines needed to do the orientations analysis are published, but also
included is the document of mixed text and code used to build this
manuscript with all of its tables and figures. It is hoped that
researchers looking at the topics discussed here will find this useful
and also that this will contribute to a growing trend in open science
publication.

\textbf{Acknowledgments}

The author thanks Tim Weaver for many productive discussions on how to
implement confidence intervals on ternary diagrams and for suggesting
the use of permutations tests to answer some statistical questions. The
author also thanks Denné Reed and Andrew Barr for pushing me into R
programming in the first place and for their continued help and
guidance. Paul Goldberg and Vera Aldeias helped push for better analysis
of these data and with their interpretation. The La Ferrassie
excavations are led by Alain Turq with Harold Dibble, Dennis Sandgathe,
Paul Goldberg, Vera Aldeias, Laurent Chiotti and the author. The
excavations were funded by the National Science Foundation, the Leakey
Foundation, Le Conseil général du département de la Dordogne, Service
régional de l'Archéologie (SRA), Institut National de la Recherche en
Archéologie Préventive (INRAP), the University of Pennsylvania Museum,
the University of Pennsylvania Research Foundation, and the Max Planck
Society. Permissions to excavate the site were granted by Jean-Jacques
Cleyet-Merle, directeur du Musée national de Préhistoire, and Nathalie
Fourment, Conservateur régional de l'Archéologie, SRA. Additional
support comes from B. Maureille, director of the PACEA at the University
of Bordeaux I. The author thanks the Max Planck Society and
Prof.~Jean-Jacques Hublin for their continued support of his research.
The author also thanks the initial PLOS One editor, Nuno Bicho, the
final PLOS One editor, John P. Hart, and three anonymous reviewers for
their help with improving this manuscript.

\textbf{Supporting Information}

\textbf{S1 File. R Markdown file used to create this document
(mcpherron\_2017.rmd).}

\textbf{S2 File. Le Ferrassie artifact orientation data set
(LF\_orientations.RDS).}

\textbf{S3 File. Lenoble and Bertran (2004) comparative data set
(Lenoble\_and\_Bertran\_2004.RDS).}

\textbf{S4 File. R source code to implement the methods presented here
(orientations.r).}

\textbf{S5 File. Aerial image of La Ferrassie (The Whole Site
(small).tif).}

\textbf{S6 File. Georeference file for aerial image of La Ferrassie (The
Whole Site (small).tfw).}

\textbf{S7 File. Citations included within this manuscript
(orientations.bib).}

\section*{References}\label{references}
\addcontentsline{toc}{section}{References}

\hypertarget{refs}{}
\hypertarget{ref-mcpherron_artifact_2005}{}
1. McPherron SJ. Artifact orientations and site formation processes from
total station proveniences. Journal of Archaeological Science. 2005;32:
1003--1014.
doi:\href{https://doi.org/doi:\%20DOI:\%2010.1016/j.jas.2005.01.015}{doi: DOI: 10.1016/j.jas.2005.01.015}

\hypertarget{ref-dibble_measurement_1987}{}
2. Dibble HL. Measurement of artifact provenience with an electronic
theodolite. Journal of Field Archaeology. 1987;14(2): 249--254.

\hypertarget{ref-kluskens_archaeological_1995}{}
3. Kluskens S. Archaeological Taphonomy of Combe-Capelle Bas from
Artifact Orientations and Density Analysis. The Middle Paleolithic Site
of Combe-Capelle Bas (France). Philadelphia: University Museum Press;
1995. pp. 199--244.

\hypertarget{ref-benn_fabric_1994}{}
4. Benn D. Fabric shape and the interpretation of sedimentary fabric
data. Journal of Sedimentary Research. 1994;A64: 910--915.

\hypertarget{ref-lenoble_fabric_2004}{}
5. Lenoble A, Bertran P. Fabric of Palaeolithic levels: Methods and
implications for site formation processes. Journal of Archaeological
Science. 2004;31: 457--469.
doi:\href{https://doi.org/10.1016/j.jas.2003.09.013}{10.1016/j.jas.2003.09.013}

\hypertarget{ref-bertran_fabric_1995}{}
6. Bertran P, Texier J-P. Fabric Analysis: Application to Paleolithic
Sites. Journal of Archaeological Science. 1995;22: 521--535.
doi:\href{https://doi.org/10.1006/jasc.1995.0050}{10.1006/jasc.1995.0050}

\hypertarget{ref-dibble_testing_1997}{}
7. Dibble HL, Chase PG, McPherron SP, Tuffreau A. Testing the Reality of
a ``Living Floor'' with Archaeological Data. American Antiquity.
1997;62: 629--651. Available: \url{http://www.jstor.org/stable/281882}

\hypertarget{ref-enloe_geological_2006}{}
8. Enloe JG. Geological processes and site structure: Assessing
integrity at a Late Paleolithic open-air site in northern France.
Geoarchaeology. 2006;21: 523--540.
doi:\href{https://doi.org/10.1002/gea.20122}{10.1002/gea.20122}

\hypertarget{ref-figueiredo_artifact_2007}{}
9. McPherron S, Dibble H. Artifact Orientations from Total Station
Proveniences. In: Figueiredo A, Velho G, editors. The World is in your
eyes. Tomar: CAAPortugal; 2007. pp. 161--166.

\hypertarget{ref-benito-calvo_sedimentological_2009}{}
10. Benito-Calvo A, Martínez-Moreno J, Jordá Pardo JF, Torre I de la,
Torcal RM. Sedimentological and archaeological fabrics in Palaeolithic
levels of the South-Eastern Pyrenees: Cova Gran and Roca dels Bous Sites
(Lleida, Spain). Journal of Archaeological Science. 2009;36: 2566--2577.
doi:\href{https://doi.org/10.1016/j.jas.2009.07.012}{10.1016/j.jas.2009.07.012}

\hypertarget{ref-chase_processes_2009}{}
11. Chase P, Debénath A, Dibble H, McPherron S. Processes of site
formation and their implications. In: Chase P, Debénath A, Dibble H,
McPherron S, editors. The Cave of Fontéchavade: Recent Excavations and
Their Paleoanthropological Implications. Cambridge: Cambridge University
Press; 2009. pp. 229--247.

\hypertarget{ref-todisco_paleoeskimo_2009}{}
12. Todisco D, Bhiry N, Desrosiers PM. Paleoeskimo site taphonomy: An
assessment of the integrity of the Tayara site, Qikirtaq Island,
Nunavik, Canada. Geoarchaeology. 2009;24: 743--791.
doi:\href{https://doi.org/10.1002/gea.20285}{10.1002/gea.20285}

\hypertarget{ref-bernatchez_taphonomic_2010}{}
13. Bernatchez JA. Taphonomic implications of orientation of plotted
finds from Pinnacle Point 13B (Mossel Bay, Western Cape Province, South
Africa). Journal of Human Evolution. 2010;59: 274--288.
doi:\href{https://doi.org/10.1016/j.jhevol.2010.07.005}{10.1016/j.jhevol.2010.07.005}

\hypertarget{ref-bertran_impact_2010}{}
14. Bertran P, Klaric L, Lenoble A, Masson B, Vallin L. The impact of
periglacial processes on Palaeolithic sites: The case of sorted
patterned grounds. Geoarchaeology and Taphonomy. 2010;214: 17--29.
doi:\href{https://doi.org/10.1016/j.quaint.2009.10.021}{10.1016/j.quaint.2009.10.021}

\hypertarget{ref-benito-calvo_trampling_2011}{}
15. Benito-Calvo A, Martínez-Moreno J, Mora R, Roy M, Roda X. Trampling
experiments at Cova Gran de Santa Linya, Pre-Pyrenees, Spain: Their
relevance for archaeological fabrics of the Upper--Middle Paleolithic
assemblages. Journal of Archaeological Science. 2011;38: 3652--3661.
doi:\href{https://doi.org/10.1016/j.jas.2011.08.036}{10.1016/j.jas.2011.08.036}

\hypertarget{ref-benito-calvo_analysis_2011}{}
16. Benito-Calvo A, Torre I de la. Analysis of orientation patterns in
Olduvai Bed I assemblages using GIS techniques: Implications for site
formation processes. Journal of Human Evolution. 2011;61: 50--60.
doi:\href{https://doi.org/10.1016/j.jhevol.2011.02.011}{10.1016/j.jhevol.2011.02.011}

\hypertarget{ref-sitzia_paleoenvironment_2012}{}
17. Sitzia L, Bertran P, Boulogne S, Brenet M, Crassard R, Delagnes A,
et al. The Paleoenvironment and Lithic Taphonomy of Shi'Bat Dihya 1, a
Middle Paleolithic Site in Wadi Surdud, Yemen. Geoarchaeology. 2012;27:
471--491.
doi:\href{https://doi.org/10.1002/gea.21419}{10.1002/gea.21419}

\hypertarget{ref-de_la_torre_application_2013}{}
18. Torre I de la, Benito-Calvo A. Application of GIS methods to
retrieve orientation patterns from imagery; a case study from Beds I and
II, Olduvai Gorge (Tanzania). Journal of Archaeological Science.
2013;40: 2446--2457.
doi:\href{https://doi.org/10.1016/j.jas.2013.01.004}{10.1016/j.jas.2013.01.004}

\hypertarget{ref-dominguez-rodrigo_testing_2013}{}
19. Domínguez-Rodrigo M, García-Pérez A. Testing the Accuracy of
Different A-Axis Types for Measuring the Orientation of Bones in the
Archaeological and Paleontological Record. PLOS ONE. 2013;8: e68955.
doi:\href{https://doi.org/10.1371/journal.pone.0068955}{10.1371/journal.pone.0068955}

\hypertarget{ref-stratford_biofabric_2013}{}
20. Stratford DJ. Biofabric analysis of palaeocave deposits. Journal of
Taphonomy. 2013;11: 21--40.

\hypertarget{ref-hovers_islands_2014}{}
21. Hovers E, Ekshtain R, Greenbaum N, Malinsky-Buller A, Nir N,
Yeshurun R. Islands in a stream? Reconstructing site formation processes
in the late Middle Paleolithic site of `Ein Qashish, northern Israel.
Opportunities, problems and future directions in the study of open-air
Middle Paleolithic sites. 2014;331: 216--233.
doi:\href{https://doi.org/10.1016/j.quaint.2014.01.028}{10.1016/j.quaint.2014.01.028}

\hypertarget{ref-zwyns_open-air_2014}{}
22. Zwyns N, Gladyshev SA, Gunchinsuren B, Bolorbat T, Flas D, Dogandžić
T, et al. The open-air site of Tolbor 16 (Northern Mongolia):
Preliminary results and perspectives. Recent advances in studies of the
late Pleistocene and Palaeolithic of Northeast Asia. 2014;347: 53--65.
doi:\href{https://doi.org/10.1016/j.quaint.2014.05.043}{10.1016/j.quaint.2014.05.043}

\hypertarget{ref-byers_flake_2015}{}
23. Byers DA, Hargiss E, Finley JB. Flake Morphology, Fluvial Dynamics,
and Debitage Transport Potential. Geoarchaeology. 2015;30: 379--392.
doi:\href{https://doi.org/10.1002/gea.21524}{10.1002/gea.21524}

\hypertarget{ref-pei_middle_2015}{}
24. Pei S, Niu D, Guan Y, Nian X, Yi M, Ma N, et al. Middle Pleistocene
hominin occupation in the Danjiangkou Reservoir Region, Central China:
Studies of formation processes and stone technology of Maling 2A site.
Journal of Archaeological Science. 2015;53: 391--407.
doi:\href{https://doi.org/10.1016/j.jas.2014.10.022}{10.1016/j.jas.2014.10.022}

\hypertarget{ref-lotter_geoarchaeological_2016}{}
25. Lotter MG, Gibbon RJ, Kuman K, Leader GM, Forssman T, Granger DE. A
Geoarchaeological Study of the Middle and Upper Pleistocene Levels at
Canteen Kopje, Northern Cape Province, South Africa. Geoarchaeology.
2016;31: 304--323.
doi:\href{https://doi.org/10.1002/gea.21541}{10.1002/gea.21541}

\hypertarget{ref-wright_approaches_2017}{}
26. Wright DK, Thompson JC, Schilt F, Cohen AS, Choi J-H, Mercader J, et
al. Approaches to Middle Stone Age landscape archaeology in tropical
Africa. Journal of Archaeological Science. 2017;77: 64--77.
doi:\href{https://doi.org/10.1016/j.jas.2016.01.014}{10.1016/j.jas.2016.01.014}

\hypertarget{ref-neudorf_comparisons_2015}{}
27. Neudorf CM, Brennand TA, Lian OB. Comparisons between macro‐and
microfabrics in a pebble‐rich, sandy till deposited by the Cordilleran
Ice Sheet. Boreas. 2015;44: 483--501.

\hypertarget{ref-ringrose_confidence_1997}{}
28. Ringrose TJ, Benn DI. Confidence regions for fabric shape diagrams.
Journal of Structural Geology. 1997;19: 1527--1536.

\hypertarget{ref-benn_random_2001}{}
29. Benn DI, Ringrose TJ. Random variation of fabric eigenvalues:
Implications for the use of a‐axis fabric data to differentiate till
facies. Earth Surface Processes and Landforms. 2001;26: 295--306.

\hypertarget{ref-ringrose_alternative_1996}{}
30. Ringrose TJ. Alternative confidence regions for canonical variate
analysis. Biometrika. 1996;83: 575--587.

\hypertarget{ref-steele_modified_2002}{}
31. Steele TE, Weaver TD. The Modified Triangular Graph: A Refined
Method for Comparing Mortality Profiles in Archaeological Samples.
Journal of Archaeological Science. 2002;29: 317--322.
doi:\href{https://doi.org/10.1006/jasc.2001.0733}{10.1006/jasc.2001.0733}

\hypertarget{ref-weaver_cross-platform_2011}{}
32. Weaver TD, Boyko RH, Steele TE. Cross-platform program for
likelihood-based statistical comparisons of mortality profiles on a
triangular graph. Journal of Archaeological Science. 2011;38:
2420--2423.
doi:\href{https://doi.org/10.1016/j.jas.2011.05.009}{10.1016/j.jas.2011.05.009}

\hypertarget{ref-larsen_fabric_2003}{}
33. Larsen NK, Piotrowski JA. Fabric pattern in a basal till succession
and its significance for reconstructing subglacial processes. Journal of
Sedimentary Research. 2003;73: 725--734.

\hypertarget{ref-r_core_team_r:_2015}{}
34. R Core Team. R: A Language and Environment for Statistical Computing
{[}Internet{]}. Vienna, Austria: R Foundation for Statistical Computing;
2015. Available: \url{https://www.R-project.org/}

\hypertarget{ref-bertran_fabric_1997}{}
35. Bertran P, Hétu B, Texier J-P, Van Steijn H. Fabric characteristics
of subaerial slope deposits. Sedimentology. 1997;44: 1--16.
doi:\href{https://doi.org/10.1111/j.1365-3091.1997.tb00421.x}{10.1111/j.1365-3091.1997.tb00421.x}

\hypertarget{ref-lenoble_solifluction-induced_2008}{}
36. Lenoble A, Bertran P, Lacrampe F. Solifluction-induced modifications
of archaeological levels: Simulation based on experimental data from a
modern periglacial slope and application to French Palaeolithic sites.
Journal of Archaeological Science. 2008;35: 99--110.
doi:\href{https://doi.org/10.1016/j.jas.2007.02.011}{10.1016/j.jas.2007.02.011}

\hypertarget{ref-guerin_multi-method_2015}{}
37. Guérin G, Frouin M, Talamo S, Aldeias V, Bruxelles L, Chiotti L, et
al. A multi-method luminescence dating of the Palaeolithic sequence of
La Ferrassie based on new excavations adjacent to the La Ferrassie 1 and
2 skeletons. Journal of Archaeological Science. 2015;58: 147--166.
doi:\href{https://doi.org/10.1016/j.jas.2015.01.019}{10.1016/j.jas.2015.01.019}

\hypertarget{ref-frouin_new_2017}{}
38. Frouin M, Guérin G, Lahaye C, Mercier N, Huot S, Aldeias V, et al.
New luminescence dating results based on polymineral fine grains from
the Middle and Upper Palaeolithic site of La Ferrassie (Dordogne, SW
France). Quaternary Geochronology. 2017;39: 131--141.
doi:\href{https://doi.org/10.1016/j.quageo.2017.02.009}{10.1016/j.quageo.2017.02.009}

\hypertarget{ref-hills_elements_1972}{}
39. Hills S. Elements of Structural Geology. New York: John Wiley \&
Sons, Inc; 1972.

\hypertarget{ref-seyfert_encyclopedia_1987}{}
40. Seyfert CK. The Encyclopedia of structural geology and plate tec
tonics {[}Internet{]}. New York: Van Nostrand Reinhold; 1987. Available:
\url{http://www.bcin.ca/Interface/openbcin.cgi?submit=submit\&Chinkey=125390}

\hypertarget{ref-woodcock_specification_1977}{}
41. Woodcock N. Specification of fabric shapes using an eigenvalue
method. Geological Society of America Bulletin. 1977;88: 1231--1236.

\hypertarget{ref-lund_circstats:_2012}{}
42. Lund S-p original by U, Agostinelli R port by C. CircStats: Circular
Statistics, from ``Topics in circular Statistics'' (2001)
{[}Internet{]}. 2012. Available:
\url{https://CRAN.R-project.org/package=CircStats}

\hypertarget{ref-ihaka_colorspace:_2016}{}
43. Ihaka R, Murrell P, Hornik K, Fisher JC, Zeileis A. Colorspace:
Color Space Manipulation {[}Internet{]}. 2016. Available:
\url{https://CRAN.R-project.org/package=colorspace}

\hypertarget{ref-stauffer_somewhere_2009}{}
44. Stauffer R, Mayr GJ, Dabernig M, Zeileis A. Somewhere over the
Rainbow: How to Make Effective Use of Colors in Meteorological
Visualizations. Bulletin of the American Meteorological Society.
2009;96: 203--216.
doi:\href{https://doi.org/10.1175/BAMS-D-13-00155.1}{10.1175/BAMS-D-13-00155.1}

\hypertarget{ref-zeileis_escaping_2009}{}
45. Zeileis A, Hornik K, Murrell P. Escaping RGBland: Selecting Colors
for Statistical Graphics. Computational Statistics \& Data Analysis.
2009;53: 3259--3270.
doi:\href{https://doi.org/10.1016/j.csda.2008.11.033}{10.1016/j.csda.2008.11.033}

\hypertarget{ref-wand_kernsmooth:_2015}{}
46. Wand M. KernSmooth: Functions for Kernel Smoothing Supporting Wand
\& Jones (1995) {[}Internet{]}. 2015. Available:
\url{https://CRAN.R-project.org/package=KernSmooth}

\hypertarget{ref-douglas_nychka_fields:_2015}{}
47. Douglas Nychka, Reinhard Furrer, John Paige, Stephan Sain. Fields:
Tools for spatial data {[}Internet{]}. Boulder, CO, USA: University
Corporation for Atmospheric Research; 2015. Available:
\url{www.image.ucar.edu/fields}

\hypertarget{ref-venables_modern_2002}{}
48. Venables WN, Ripley BD. Modern Applied Statistics with S
{[}Internet{]}. Fourth. New York: Springer; 2002. Available:
\url{http://www.stats.ox.ac.uk/pub/MASS4}

\hypertarget{ref-urbanek_tiff:_2013}{}
49. Urbanek S. Tiff: Read and write TIFF images {[}Internet{]}. 2013.
Available: \url{https://CRAN.R-project.org/package=tiff}

\hypertarget{ref-wickham_dplyr:_2016}{}
50. Wickham H, Francois R. Dplyr: A Grammar of Data Manipulation
{[}Internet{]}. 2016. Available:
\url{https://CRAN.R-project.org/package=dplyr}

\hypertarget{ref-wickham_reshaping_2007}{}
51. Wickham, Hadley. Reshaping data with the reshape package. Journal of
Statistical Software. 2007;21. Available:
\url{http://www.jstatsoft.org/v21/i12/paper}

\hypertarget{ref-stodden_knitr:_2014}{}
52. Xie Y. Knitr: A Comprehensive Tool for Reproducible Research in R.
In: Stodden V, Leisch F, Peng RD, editors. Implementing Reproducible
Computational Research. Chapman; Hall/CRC; 2014. Available:
\url{http://www.crcpress.com/product/isbn/9781466561595}

\hypertarget{ref-xie_dynamic_2015}{}
53. Xie Y. Dynamic Documents with R and knitr {[}Internet{]}. 2nd ed.
Boca Raton, Florida: Chapman; Hall/CRC; 2015. Available:
\url{http://yihui.name/knitr/}

\hypertarget{ref-xie_knitr:_2016}{}
54. Xie Y. Knitr: A General-Purpose Package for Dynamic Report
Generation in R {[}Internet{]}. 2016. Available:
\url{http://yihui.name/knitr/}

\hypertarget{ref-dahl_xtable:_2016}{}
55. Dahl DB. Xtable: Export Tables to LaTeX or HTML {[}Internet{]}.
2016. Available: \url{https://CRAN.R-project.org/package=xtable}

\hypertarget{ref-boettiger_knitcitations:_2015}{}
56. Boettiger C. Knitcitations: Citations for 'Knitr' Markdown Files
{[}Internet{]}. 2015. Available:
\url{https://CRAN.R-project.org/package=knitcitations}

\hypertarget{ref-hoffman_reproducibility:_2016}{}
57. Hoffman JI. Reproducibility: Archive computer code with raw data.
Nature. 2016;534: 326--326. Available:
\url{http://dx.doi.org/10.1038/534326d}


\end{document}
